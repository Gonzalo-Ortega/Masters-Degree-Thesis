\section{Preliminaries (to do work)}
The contents of this chapter are based on \cite{nanda}, \cite{polterovich} and \cite{wang}.

\begin{definition}[Graded ring]
    
\end{definition}

\begin{definition}[Graded ideal]
    
\end{definition}

\begin{definition}[Graded moudule]
    
\end{definition}

\begin{definition}[Persistance module, finite type]
    $ (V, \pi) $, where $ V = \{V_t\}_{t \in \mathbb R} $ is a collection of finite dimensional vector spaces over a field $ \mathbb F $, and $ \pi = \{ \pi_{s \leq t} \} $ is a collection of linear maps $ \pi_{s \leq t}: V_s \rightarrow V_t $.
\end{definition}

\begin{definition}[Barcode]

\end{definition}

\begin{definition}[$\delta$-interleaving]
    
\end{definition}

\begin{definition}[Interleaving distance]
    
\end{definition}

\begin{definition}[$\delta$-matching]
    
\end{definition}

\begin{definition}[Bottleneck distance]
    
\end{definition}

\newpage
\section{Structure Theorem}
\begin{fact}[Structure theorem for finitely generated modules over a principal ideal domain] \label{structure-pure}
    Let $ M $ be a  finitely generated module over a principal ideal domain. There exist a finite sequence of proper ideals $ (d_1) \supseteq (d_2) \supseteq \dots \supseteq (d_n) $ such that
    $$
        M \cong \bigoplus_{i=1}^n R / (d_i).
    $$

\end{fact}

\begin{proposition}
    An ideal $ I \subseteq R $ is graded if and only if it is generated by homogeneous elements.
\end{proposition}

\begin{theorem}[Structure]
    Let $ (V, \pi) $ be a persistence module. There exist a finite set $ \operatorname{bar}(V, \pi) $ of intervals and a function $ \mu : \operatorname{bar}(V, \pi) \longrightarrow \mathbb N $ and there is a unique direct sum decomposition
    $$
        (V, \pi) \cong \bigoplus_{i=1}^N (I_i, c_i)^{m_i}.
    $$
\end{theorem}
\begin{proof}
    $ (V, \pi) $ is of finite type, so it is a finite $ \mathbb F[x] $-module. As $ \mathbb F $ is a field, $ \mathbb F[x] $ is a principal ideal domain, therefore, $ V $ is a finitely generated module over a principal ideal domain. Using Fact \ref{structure-pure} $ V $ can be decompose in the direct sum of its free and torsion subgroups.
\end{proof}

\newpage
\section{Stability Theorem}
\begin{lemma}
    
\end{lemma}

\begin{theorem}[Stability]
    Given two persistence modules $ (V, \pi) $, $ (W, \phi) $, we have
    $$
        d_{int} ((V, \pi), (W, \phi)) = d_{bot} (\operatorname{bar}(V, \pi), \operatorname{bar}(W, \phi)).
    $$
\end{theorem}
